%!TEX output_directory = bin

\documentclass[a4papert]{article}

\input{/home/aghiles/Aghiles/Redaction/setup/setup}
\input{/home/aghiles/Aghiles/Redaction/setup/setup_paper}

\usepackage[papersize={8.5in,12in}, left=0.9in, right=0.9in, top=1in, bottom=1in]{geometry}

%\title{Projet PFE: Air pollution monitoring system using LoRa communication}
\title{\textbf{Projet PFE:} Smart Construction}
\date{Septembre 2019}

\begin{document}
%\setcounter{page}{0}
\maketitle
%\thispagestyle{fancy}


%\section{Introduction}

\section{Design a low cost pollution application}

In order to develop pollution sensors to predict pollution based on data sent by sensors network implanted in a construction site\cite{alvear_architecture_2016,rosmiati_air_2019}.
We want to build an air pollution monitoring system which become an essential requirement for cities worldwide.
Currently,
	the most extended way to monitor air pollution and noise pollution is via fixed monitoring stations\cite{paredes-parra_alternative_2019},
	which are expensive and hard to install.
To solve this problem,
	we propose a Web Sensor Data Processing Engine,
	a solution to monitor air pollution through mobile sensors.
It will be deployed with off-the-shelf hardware such as Arduino, Zolertia and Raspberry Pi devices.

This application analyzes,
	represents and displays the correlations between traffic,
	speed of pollution travel and exposure.
Users will be asked to enable GPS on their device to get their current position in order to receive data of the closest sensors.
This application collects air pollution using embedded sensors and transfers the captured data to the NoSQL database of your choice (MongoDB, ...).
A web interface should displays to the user the air pollution levels in real time.
The application also stores the different pollution traces to the Edge-based server via Wireless communication to analyze the pollution distribution.
% The Edge server uses the uploaded data,
% 	together with highly-accurate data made available by the existing air monitoring infrastructure,
% 	to create detailed pollution distribution maps using kriging-based spatial prediction techniques.
The system should also provide an alert mechanism which notifies the different level of authorities through email and SMS in case of any issues.
A web page provides the interface to the residents and to the authorities to gauge the air quality after analyzing the data using the prediction algorithm.

\section{LoRa}

The use of Low Power Wide Area Network (LPWAN) technology using LoRa as a transceiver medium is expected to overcome the distance problem in transmit data testing,
	the main advantage of LoRa is the low use of power when transmitting information.
Thus,
	the life span of LoRa batteries is around 10 years.
With some advantages possessed by LoRa,
	this system can support IoT communication infrastructure.
The sensors attached with a micro controller in the LoRa module will communicate to the Edge environment through the LoRa gateway.


\section{Tools and background}

\begin{itemize}
	\item Python, Java, PHP and Html.
	\item Strong knowledge on web applications
	\item Knowledge on learning algorithms.
\end{itemize}


\printbibliography
%The sensor captures information on:
%\Itemize{
%	\item PM
%	\item Humidity
%	\item Temperature
%	\item Pressure
%	\item GPS
%}



%\printbibliography

\end{document}


