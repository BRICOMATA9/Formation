%!TEX output_directory = bin

\documentclass[a4paper]{article}


\input{/home/aghiles/Aghiles/Redaction/setup/setup_paper}
\input{/home/aghiles/Aghiles/Redaction/setup/setup}

\usepackage[papersize={8.5in,11in}, left=0.9in, right=0.9in, top=1in, bottom=1in]{geometry}

%\title{Projet PFE: Air pollution monitoring system using LoRa communication}
\title{\textbf{Projet PPE:} Prédiction de la pollution dans une rue selon l’apprentissage}
\date{Septembre 2019}

\begin{document}
%\setcounter{page}{0}
\maketitle
%\thispagestyle{fancy}


In order to evaluate traffic congestion on urban roads and study their impact on air quality,
	we propose to develop an IoT application based on traffic management.
Successful development of effective real-time traffic management and information systems requires high quality traffic information in real-time \cite{lopes_traffic_2010}.
This project presents the state-of-the-art of traffic management for data pre-processing and cleaning for real-time applications.
Such application is extremely important to evaluate the impact of road traffic congestion on the environment,
	therefore,
	the reliability of information and outputs derived from data fusion and processing is extremely important to provide knowledge of the air quality at each time.
Mitigating traffic congestion on urban roads depends on our ability to foresee road usage and traffic conditions pertaining to the collective behavior of drivers \cite{wang_predictability_2015}.
The goal of this project is to simulate traffic with SUMO \cite{behrisch_sumo_2011} and study the relationship between air pollution and traffic congestion.


\section{Tools and background}

\begin{itemize}
	\item SUMO and ns3.
	\item Knowledge on learning algorithms.
\end{itemize}

\printbibliography


%\printbibliography

\end{document}


