\environment miseenpage3
\environment commandes
%\environment libertine % ne marche plus avec context minimal
\chardef\characteralilgnmentmode=2
%\setuplayout[grid=no] \showgrid

\starttext
%\showframe
    {\tfb Les fiches de Bébert}
\blank[2*big]
\startalignment[center]
  \blank[5*big]
    {\tfd \CONTEXT, les tableaux}
  \blank[3*medium]
    {\tfa Bertrand {\sc Masson}}
  \blank[2*medium]
    {\tfa \date}
  \blank[3*medium]
\stopalignment
\bgroup
%\tfx
L'esprit des fiches de Bébert sur \CONTEXT\ est différent de celles concernant \LaTeX. J'ai décidé de me mettre à \CONTEXT. Comme à chaque foi que je me lance dans de nouveaux apprentissages je prend des notes, beaucoup de notes. Et dans le cas de \CONTEXT\ encore plus car la documentation en français est quasi inexistante et mon anglais indigent. Je me suis rendu compte que la réalisation des fiches sur \LaTeX\ m'avait permis de mieux clarifier et structurer mes connaissances de \LaTeX\ et finalement d'être plus productif. Donc réaliser des fiches sur \CONTEXT\ devrait avoir le même résultat. Enfin pour moi, pour vous je ne suis pas sûr, car la qualité des fiches risque d'être moins bonne. Je ne maîtrise pas \CONTEXT\ (et peu l'anglais) et donc ces fiches sont un état de ma compréhension de la bête. Il risque d'y avoir des erreurs. Vous voilà prévenu. De plus ces fiches sont loin d'être exhaustives, car pour l'instant je survol, plus que j'approfondis, pour me faire une idée et voir si \CONTEXT\ va répondre aux problèmes de mise en page que j'ai rencontré avec \LaTeX. 
\startRemarque
N'hésitez pas, si vous rencontrez des erreurs, des incompréhensions, d'autres manières de faire plus subtiles de m'en faire part. Cela aura le double objectif de me faire progresser (et je le vaux bien) et d'améliorer la qualité de ces fiches. On obtiendra un début de documentation en français sur \CONTEXT\ (et vous le valez bien).
\stopRemarque
Ces fiches sont réalisées avec \CONTEXT\, elles sont donc le reflet de mes connaissances, leur qualité esthétique devrait logiquement s'améliorer avec le temps. Elles me serviront aussi de laboratoire et donc devraient passer par des phases plus ou moins farfelues comme ci-dessus. 

Les \quote{fiches de Bébert}  sur \CONTEXT\ sont basées sur l'abondante documentation fournie par \from[garden] et \from[pragma]. Certains passages ou exemples sont des traductions de cette documentation.
\startAlert
N'hésitez pas, non plus a me faire des remarques sur l'orthographe, car si mon anglais est indigent, mon orthographe française est pitoyable.
\stopAlert
\egroup
\page[yes]
\section{Introduction}
Il y a plusieurs environnements pour créer un tableau avec  \CONTEXT.

\startitemize[2]
\item Tabulate : pour créer des tableaux simples, intégrés au texte et sans filets verticaux ;
\item TABLE  : appelé aussi tableaux naturels (natural tables) ou tableau HTML car la syntaxe est proche de celle du HTML. C'est cet environnement qui est recommandé ; 
\item Linetable : est un environnement expérimental pour les tableaux multi-pages, il n'existe pas de documentation.
\item Table : l'ancienne méthode, qu'il n'est plus recommandée d'utiliser, mais qui est celle de décrite dans les manuels \CONTEXT\ ;
\item Tables : l'ancienne méthode pour les tableaux multi-pages, qu'il n'est plus recommandée d'utiliser, mais qui est celle de décrite dans les manuels \CONTEXT\ ;
\stopitemize

Bien entendu dans cette fiche nous allons étudier l'environnement TABLE. 

\section{L'environnement TABLE}
Toutes les commandes pour construire un tableau sont des environnements, elles commencent toutes par un {\it begin} (abrégé par {\tt b})  et s'achève par un {\it end} ({\tt e}). Donc pour créer un tableau tu utilises :
\starttyping[option=TEX]
\bTABLE
....
\eTABLE
\stoptyping

Pour une colonne  :
\starttyping[option=TEX]
\bTD .... \eTD
\stoptyping
Une ligne par :
\starttyping[option=TEX]
\bTR .... \eTR
\stoptyping
Des colonnes à en-tête par :
\starttyping[option=TEX]
\bTH .... \eTH
\stoptyping
H pour Head, met le texte de la ligne en gras.

{\red Attention} les majuscules sont obligatoires.
Voici un premier tableau :
\starttyping[option=TEX]
\bTABLE
\bTR \bTH entête 1\eTH \bTH entête 2 \eTH \bTH entête 3 \eTH \eTR
\bTR \bTD colonne 1 ligne 1\eTD \bTD c2 l1 \eTD \bTD c3 l1 \eTD \eTR
\bTR \bTD colonne 1 ligne 2\eTD \bTD c2 l2 \eTD \bTD c3 l2 \eTD \eTR
\eTABLE
\stoptyping
\bTABLE
\bTR \bTH entête 1\eTH \bTH entête 2 \eTH \bTH entête 3 \eTH \eTR
\bTR \bTD colonne 1 ligne 1\eTD \bTD c2 l1 \eTD \bTD c3 l1 \eTD \eTR
\bTR \bTD colonne 1 ligne 2\eTD \bTD c2 l2 \eTD \bTD c3 l2 \eTD \eTR
\eTABLE

\section{Fusionner les cellules horizontalement}

Tu utilises l'option {\tt nc=}{\it n} où {\it n} est le nombre de cellules fusionnées

\starttyping[option=TEX]
\bTABLE
\bTR \bTD colonne 1 ligne 1\eTD \bTD[nc=2] c2 l1 et c3 l1 \eTD \eTR
\bTR \bTD colonne 1 ligne 2\eTD \bTD c2 l2 \eTD \bTD c3 l2 \eTD \eTR
\bTR \bTD[nc=3] toutes les cellules fusionnées \eTD \eTR
\eTABLE
\stoptyping

\bTABLE
\bTR \bTD colonne 1 ligne 1\eTD \bTD[nc=2] c2 l1 et c3 l1 \eTD \eTR
\bTR \bTD colonne 1 ligne 2\eTD \bTD c2 l2 \eTD \bTD c3 l2 \eTD \eTR
\bTR \bTD[nc=3] toutes les cellules fusionnées \eTD \eTR
\eTABLE

\section{Fusionner les cellules verticalement}

Tu utilises l'option {\tt nr=}{\it n} où {\it n} est le nombre de cellules fusionnées

\starttyping[option=TEX]
\bTABLE
\bTR \bTD colonne 1 ligne 1\eTD \bTD c2 l1\eTD	\bTD[nr=3] c3 l1 \eTD 	\eTR
\bTR \bTD[nr=2] colonne 1 ligne 2 et 3	 \eTD	\bTD c2 l2 \eTD  		\eTR
\bTR 							\bTD c2 l3 \eTD		\eTR
\eTABLE
\stoptyping

\bTABLE
\bTR \bTD colonne 1 ligne 1\eTD \bTD c2 l1\eTD	\bTD[nr=3] c3 l1 \eTD 	\eTR
\bTR \bTD[nr=2] colonne 1 ligne 2 et 3	 \eTD	\bTD c2 l2 \eTD  		\eTR
\bTR 										  		\bTD c2 l3 \eTD		\eTR
\eTABLE

Comme tu peux le voir tu n'es pas obligé comme avec \LaTeX\ de créer des cellules vides dans les colonnes fusionnées.

\section{Configurer le tableau}
La configuration d'un tableau ce fait à plusieurs niveaux. Tu peux configurer l'ensemble du tableau, une seule ou un ensemble de lignes, une seule ou un ensemble de colonnes, ou bien  une seule ou un ensemble de cellules. Tu peux également avoir des options qui s'appliquent à l'ensemble du tableaux et d'autre à des éléments particuliers.
\subsection{Les options de configuration}
\subsubsection{L'alignement du texte}
C'est l'option {\tt align} qui peux prendre les valeurs : {\tt left middle right} pour aligner le texte respectivement à gauche, au centre et à droite. 

La valeur {\tt lohi} permet de centrer verticalement le contenu de la cellule.

  Si tu utilises deux valeurs, tu les encadre de \{ \}. Par exemple pour centrer le contenu horizontalement et verticalement : 
\starttyping[option=TEX]
align={middle,lohi}
\stoptyping

\subsubsection{La largeur et la hauteur des cellules}
{\tt width=dimension} : règle la largeur des colonnes.
\blank[small]
{\tt height=dimension} : règle la hauteur des lignes.

\subsubsection{Les traits}
{\tt frame=on/off} : par défaut {\tt frame} vaut {\tt on} et donc un cadre entoure la cellule.
\blank[small]
{\tt topframe=on/off} : le trait du haut de la cellule.
\blank[small]
{\tt bottomframe=on/off} : le trait du bas de la cellule.
\blank[small]
{\tt leftframe=on/off} : le trait de gauche de la cellule.
\blank[small]
{\tt rightframe=on/off} : le trait de droite de la cellule.
\blank[small]
{\tt rulethickness=dimension} : l'épaisseur des traits entourant la cellule.
Pour pouvoir utiliser {\tt topframe bottomframe leftframe rightframe} il faut au préalable mettre {\tt frame=off}.
\subsubsection{La couleur}
Avant toute chose, dès que tu veux utiliser la couleur dans un document avec \CONTEXT, tu dois ajouter dans ton préambule :
\starttyping[option=TEX]
\setupcolors[state=start]
Les options pour la couleur dans un tableau sont :


\stoptyping
{\tt color=nom de la couleur} :  colorie le texte.
\blank[small]
{\tt background=color} : Attention c'est le mot {\tt color} et pas le nom de la couleur, c'est en gros un commutateur qui passe le tableau en mode couleur. On verra plus tard que {\tt background} peut prendre d'autre valeur.
\blank[small]
{\tt backgroundcolor=nom de la couleur} : ici tu indiques la couleur de fond de la cellule.
\blank[small]
{\tt framecolor=nom de la couleur} : la couleur des filets.

Pour utiliser des niveaux de gris :

{\tt background=screen}

et

{\tt backgroundscreen=nombre}, avec nombre compris entre 0 (noir) et 1 (blanc), donc 0.25 donnera un gris foncé et 0.85 un gris clair.

\subsubsection{Changer de fonte}
{\tt style =   normal  bold  slanted  boldslanted  type  cap  small...}

Tu peux également employer des commandes comme : \type{style=\tfx\it}

\subsection{Configurer tout le tableau}
Tu as deux possibilités. La première au niveau de \type{\bTABLE}. Par exemple pour avoir un tableau avec toutes les colonnes d'une largeur de 4\,cm et les filets bleus :
\starttyping[option=TEX]
\bTABLE[width=4cm,framecolor=blue]
\bTR \bTD colonne 1 ligne 1\eTD \bTD c2 l1 \eTD \bTD c3 l1 \eTD \eTR
\bTR \bTD colonne 1 ligne 2\eTD \bTD c2 l2 \eTD \bTD c3 l2 \eTD \eTR
\eTABLE
\stoptyping
\bTABLE[width=4cm,framecolor=blue]
\bTR \bTD colonne 1 ligne 1\eTD \bTD c2 l1 \eTD \bTD c3 l1 \eTD \eTR
\bTR \bTD colonne 1 ligne 2\eTD \bTD c2 l2 \eTD \bTD c3 l2 \eTD \eTR
\eTABLE
Tu peux également utiliser 
\starttyping[option=TEX]
\setupTABLE[width=4cm,framecolor=blue]
\stoptyping
que tu places dans l'entête pour affecter tous les tableaux de ton document ou juste avant le tableau à modifier. Tu limite sa portée le cas échéant avec \{ \} ou \type{\bgroup ... \egroup}

\subsection{Configurer une ou des ligne(s)}
\starttyping[option=TEX]
\setupTABLE[row][n][options]
\stoptyping
{\tt row} ligne en anglais que tu peux abréger en {\tt [r]}
\blank[small]
{\tt [n]} peux prendre plusieurs valeurs. Tout d'abord le numéro de la ligne que tu veux modifier. Si tu veux en affecter plusieurs tu sépares leur numéro par une virgule : {\tt [3]} modifie la troisième ligne et {\tt [2,7,8]} affecte les deuxième, septième et huitième.

{\tt [r][first]} : modifie la première ligne.
\blank[small]
{\tt [r][last]} : modifie la dernière ligne.
\blank[small]
{\tt [r][odd]} : modifie toutes les lignes impaires.
\blank[small]
{\tt [r][even]} : modifie toutes les lignes paires.
\blank[small]
Tu peux utiliser autant de \type{\setupTABLE} que nécessaire.
\starttyping[option=TEX]
{\setupTABLE[width=4cm]
\setupTABLE[r][even][background=color,backgroundcolor=red]
\setupTABLE[r][first][background=screen,backgroundscreen=0.85]
\setupTABLE[r][3][rulethickness=2pt]

\bTABLE
\bTR \bTD colonne 1 ligne 1\eTD \bTD c2 l1 \eTD \bTD c3 l1 \eTD \eTR
\bTR \bTD colonne 1 ligne 2\eTD \bTD c2 l2 \eTD \bTD c3 l2 \eTD \eTR
\bTR \bTD colonne 1 ligne 3\eTD \bTD c2 l2 \eTD \bTD c3 l3 \eTD \eTR
\bTR \bTD colonne 1 ligne 4\eTD \bTD c2 l2 \eTD \bTD c3 l4 \eTD \eTR
\bTR \bTD colonne 1 ligne 5\eTD \bTD c2 l2 \eTD \bTD c3 l5 \eTD \eTR
\eTABLE}
\stoptyping
{\setupTABLE[width=4cm]
\setupTABLE[r][even][background=color,backgroundcolor=red]
\setupTABLE[r][first][background=screen,backgroundscreen=0.85]
\setupTABLE[r][3][rulethickness=2pt]

\bTABLE
\bTR \bTD colonne 1 ligne 1\eTD \bTD c2 l1 \eTD \bTD c3 l1 \eTD \eTR
\bTR \bTD colonne 1 ligne 2\eTD \bTD c2 l2 \eTD \bTD c3 l2 \eTD \eTR
\bTR \bTD colonne 1 ligne 3\eTD \bTD c2 l2 \eTD \bTD c3 l3 \eTD \eTR
\bTR \bTD colonne 1 ligne 4\eTD \bTD c2 l2 \eTD \bTD c3 l4 \eTD \eTR
\bTR \bTD colonne 1 ligne 5\eTD \bTD c2 l2 \eTD \bTD c3 l5 \eTD \eTR
\eTABLE}
\subsection{Configurer une ou des colonne(s)}
C'est le même principe que pour les lignes, avec {\tt [column]} abrégé en {\tt [c]} et les mêmes options.
\starttyping[option=TEX]
{\setupTABLE[width=3.5cm]
\setupTABLE[c][even][background=color,backgroundcolor=red]
\setupTABLE[c][first][background=screen,backgroundscreen=0.85]
\setupTABLE[c][3][rulethickness=2pt]

\bTABLE
\bTR \bTD colonne 1 ligne 1\eTD \bTD c2 l1 \eTD \bTD c3 l1 \eTD \bTD c4 l1 \eTD\eTR
\bTR \bTD colonne 1 ligne 2\eTD \bTD c2 l2 \eTD \bTD c3 l2 \eTD \bTD c4 l2 \eTD\eTR
\bTR \bTD colonne 1 ligne 3\eTD \bTD c2 l2 \eTD \bTD c3 l3 \eTD \bTD c4 l3 \eTD\eTR
\bTR \bTD colonne 1 ligne 4\eTD \bTD c2 l2 \eTD \bTD c3 l4 \eTD \bTD c4 l4 \eTD\eTR
\bTR \bTD colonne 1 ligne 5\eTD \bTD c2 l2 \eTD \bTD c3 l5 \eTD \bTD c4 l5 \eTD\eTR
\eTABLE}

\stoptyping

{\setupTABLE[width=3.5cm]
\setupTABLE[c][even][background=color,backgroundcolor=red]
\setupTABLE[c][first][background=screen,backgroundscreen=0.85]
\setupTABLE[c][3][rulethickness=2pt]

\bTABLE
\bTR \bTD colonne 1 ligne 1\eTD \bTD c2 l1 \eTD \bTD c3 l1 \eTD \bTD c4 l1 \eTD\eTR
\bTR \bTD colonne 1 ligne 2\eTD \bTD c2 l2 \eTD \bTD c3 l2 \eTD \bTD c4 l2 \eTD\eTR
\bTR \bTD colonne 1 ligne 3\eTD \bTD c2 l2 \eTD \bTD c3 l3 \eTD \bTD c4 l3 \eTD\eTR
\bTR \bTD colonne 1 ligne 4\eTD \bTD c2 l2 \eTD \bTD c3 l4 \eTD \bTD c4 l4 \eTD\eTR
\bTR \bTD colonne 1 ligne 5\eTD \bTD c2 l2 \eTD \bTD c3 l5 \eTD \bTD c4 l5 \eTD\eTR
\eTABLE}


\subsection{Configurer une cellule}
Chaque cellule d'un tableau est identifiée par son numéro de colonne et son numéro de ligne. 

Tu peux la configurer par :
\starttyping[option=TEX]
\setupTABLE[numéro de colonne][numéro de ligne][options]
\stoptyping

\starttyping[option=TEX]
\setupTABLE[3][2][background=color,backgroundcolor=red]
\bTABLE
\bTR \bTD colonne 1 ligne 1\eTD \bTD c2 l1 \eTD \bTD c3 l1 \eTD \bTD c4 l1 \eTD\eTR
\bTR \bTD colonne 1 ligne 2\eTD \bTD c2 l2 \eTD \bTD c3 l2 \eTD \bTD c4 l2 \eTD\eTR
\bTR \bTD colonne 1 ligne 3\eTD \bTD c2 l2 \eTD \bTD c3 l3 \eTD \bTD c4 l3 \eTD\eTR
\eTABLE
\stoptyping

{\setupTABLE[3][2][background=color,backgroundcolor=red]
\bTABLE
\bTR \bTD colonne 1 ligne 1\eTD \bTD c2 l1 \eTD \bTD c3 l1 \eTD \bTD c4 l1 \eTD\eTR
\bTR \bTD colonne 1 ligne 2\eTD \bTD c2 l2 \eTD \bTD c3 l2 \eTD \bTD c4 l2 \eTD\eTR
\bTR \bTD colonne 1 ligne 3\eTD \bTD c2 l2 \eTD \bTD c3 l3 \eTD \bTD c4 l3 \eTD\eTR
\eTABLE}
\blank
Tu peux également intervenir directement au niveau de la cellule :

\starttyping[option=TEX]
\bTABLE
\bTR \bTD c1 l1\eTD \bTD[rulethickness=2pt] c2 l1 \eTD \bTD c3 l1 \eTD \bTD c4 l1 \eTD\eTR
\bTR \bTD c1 l2\eTD \bTD[color=red] c2 l2 \eTD \bTD[nr=2,align=lohi] c3 l2 \eTD \bTD c4 l2 \eTD\eTR
\bTR \bTD c1 l3\eTD \bTD c2 l2 \eTD  \bTD[color=blue] c4 l3 \eTD\eTR
\eTABLE
\stoptyping
\bTABLE
\bTR \bTD c1 l1\eTD \bTD[rulethickness=2pt] c2 l1 \eTD \bTD c3 l1 \eTD \bTD c4 l1 \eTD\eTR
\bTR \bTD c1 l2\eTD \bTD[color=red] c2 l2 \eTD \bTD[nr=2,align=lohi] c3 l2 \eTD \bTD c4 l2 \eTD\eTR
\bTR \bTD c1 l3\eTD \bTD c2 l2 \eTD  \bTD[color=blue] c4 l3 \eTD\eTR
\eTABLE

\section{Aligner le texte d'une colonne sur un caractère}
Si tu veux par exemple aligner tous les nombres d'une colonne sur la décimale, il faut utiliser 2 options :

{\tt alignmentcharacter} qui définit le caractère sur lequel se fait l'alignement, par exemple pour la virgule \type{alignmentcharacter={,}}.

{\tt aligncharacter} qui est un commutateur et prend les valeurs {\tt yes} ou {\tt no}.
\starttyping[option=TEX]
\setupTABLE[c][2][alignmentcharacter={,},aligncharacter=yes]
\setupTABLE[c][1][align=right]
\setupTABLE[r][1][style=bold]
\bTABLE
\bTR \bTD Nom   \eTD \bTD Valeur \eTD \eTR
\bTR \bTD Raoul   \eTD \bTD 2,725 \eTD \eTR
\bTR \bTD Simone   \eTD \bTD 125,58 \eTD \eTR
\bTR \bTD Ginette   \eTD \bTD 1,2 \eTD \eTR
\eTABLE
\stoptyping

{\setupTABLE[c][2][alignmentcharacter={,},aligncharacter=yes]
\setupTABLE[c][1][align=right]
\setupTABLE[r][1][style=bold]
\bTABLE
\bTR \bTD Nom   \eTD \bTD Valeur \eTD \eTR
\bTR \bTD Raoul   \eTD \bTD 2,725 \eTD \eTR
\bTR \bTD Simone   \eTD \bTD 125,58 \eTD \eTR
\bTR \bTD Ginette   \eTD \bTD 1,2 \eTD \eTR
\eTABLE}
\section{Des options piochées à droite et à gauche}
Je n'ai pas trouvé dans la documentation une description de \type{\setupTABLE}. Mais voici une une liste d'options que j'ai pioché au gré de la lecture de différents exemples.
\subsection{{\tt distance}}

{\tt distance=dimension} distance entre les colonnes, pour comprendre voici un exemple :
\starttyping[option=TEX]
\bTABLE[distance=3em]
 \bTR \bTD Un   \eTD \bTD Deux   \eTD \bTD Trois \eTD \eTR
 \bTR \bTD One   \eTD \bTD Two   \eTD \bTD Three \eTD \eTR
\eTABLE
\stoptyping
\bTABLE[distance=3em]
 \bTR 
    \bTD Un   \eTD 
    \bTD Deux   \eTD
    \bTD Trois \eTD
 \eTR
 \bTR 
    \bTD One   \eTD 
    \bTD Two   \eTD
    \bTD Three \eTD
 \eTR
\eTABLE

J'ai également vu {\tt columndistance} qui semble équivalent.
\subsection{{\tt leftmargindistance}}
 {\tt leftmargindistance=dimension} distance du tableau par rapport à la marge.
\starttyping[option=TEX]
\bTABLE[leftmargindistance=3cm]
 \bTR \bTD Un   \eTD \bTD Deux   \eTD \bTD Trois \eTD \eTR
 \bTR \bTD One   \eTD \bTD Two   \eTD \bTD Three \eTD \eTR
\eTABLE
\stoptyping
\bTABLE[leftmargindistance=3cm]
 \bTR 
    \bTD Un   \eTD 
    \bTD Deux   \eTD
    \bTD Trois \eTD
 \eTR
 \bTR 
    \bTD One   \eTD 
    \bTD Two   \eTD
    \bTD Three \eTD
 \eTR
\eTABLE

Il existe également {\tt rightmargindistance} mais je n'ai pas compris son utilité.

\subsection{{\tt option}}
Je n'ai trouvé qu'une seule valeur pour {\tt option} c'est {\tt stretch} qui force le tableau à occuper toute la largeur du texte.
\starttyping[option=TEX]
\bTABLE[option=stretch]
 \bTR \bTD Un   \eTD \bTD Deux   \eTD \bTD Trois \eTD \eTR
 \bTR \bTD One   \eTD \bTD Two   \eTD \bTD Three \eTD \eTR
\eTABLE
\stoptyping
\bTABLE[option=stretch]
 \bTR 
    \bTD Un   \eTD 
    \bTD Deux   \eTD
    \bTD Trois \eTD
 \eTR
 \bTR 
    \bTD One   \eTD 
    \bTD Two   \eTD
    \bTD Three \eTD
 \eTR
\eTABLE
\subsection{{\tt offset}}
Règle la distance entre les filets et le texte.

{\tt offset=dimension}

\starttyping[option=TEX]
\bTABLE[offset=2em]
 \bTR \bTD Un   \eTD \bTD Deux   \eTD \bTD Trois \eTD \eTR
 \bTR \bTD One   \eTD \bTD Two   \eTD \bTD Three \eTD \eTR
\eTABLE
\stoptyping
\bTABLE[offset=2em]
 \bTR 
    \bTD Un   \eTD 
    \bTD Deux   \eTD
    \bTD Trois \eTD
 \eTR
 \bTR 
    \bTD One   \eTD 
    \bTD Two   \eTD
    \bTD Three \eTD
 \eTR
\eTABLE
\section{Conclusion}
Voici une première approche de la fabrication de tableaux avec \CONTEXT. La création de tableaux semble quand même plus facile que sous \LaTeX, où il faut multiplier les packages pour arriver à un bon résultat. Pour l'instant ce qui se trouve dans cette fiche m'a permis de réaliser tous les tableaux dont j'avais besoin, mais ils n'étaient pas très compliqués. 

Les documents suivants avec leurs nombreux exemples peuvent t'aider :

\from[tableauNaturel]

\from[tableauNaturel2]

\stoptext